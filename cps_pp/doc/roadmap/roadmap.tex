

%------------------------------------------------------------------
% $Id: roadmap.tex,v 1.4 2004/08/18 11:57:35 zs Exp $
%------------------------------------------------------------------
%Anj: EPCC: e-mail: a.jackson@epcc.ed.ac.uk
%
% For best results, this latex file should be compiled using pdflatex.
% However it will also compile under normal latex, if you prefer.
%
%------------------------------------------------------------------
\documentclass[12pt]{article}

% importing other useful packages:
\usepackage{times}
\usepackage{fullpage}
\usepackage{graphicx}
\usepackage{longtable}
\usepackage{tabularx}
% color for the pdf links:
\usepackage{color}
\definecolor{darkblue}{rgb}{0.0,0.0,0.5}
% for conditional latex source:
\usepackage{ifthen}
% pdftex specifications, these are only included if we are using pdflatex:
\providecommand{\pdfoutput}{0}
\ifthenelse{\pdfoutput = 0}{
% Not PDF:
\usepackage{html}
\newcommand{\hreff}[2]{\htmladdnormallink{#2}{#1}}
}{
% PDF: hyperref for pdf with full linkages:
\usepackage[
pagebackref,
hyperindex,
hyperfigures,
colorlinks,
linkcolor=darkblue,
citecolor=darkblue,
pagecolor=darkblue,
urlcolor=blue,
%bookmarksopen,
pdfpagemode=None,
%=UseOutlines,
pdfhighlight={/I},
pdftitle={CPS: Development Roadmap. $Revision: 1.4 $ - $Date: 2004/08/18 11:57:35 $.},
pdfauthor={A.N. Jackson \& S. Booth},
plainpages=false
]{hyperref}
}

% CVS tag shorthand:
\newcommand{\ukqcdtag}[1]{{\small{UKQCD CVS Repository Tag: }{\bf{\tt{#1}}}}}

% Code style commands:
\newcommand{\cls}[1]{{\bf #1}}            % Classes
\newcommand{\struct}[1]{{\bf #1}}         % Structs
\newcommand{\cde}[1]{{\tt #1}}            % Code fragments

% document style modifications:
\setlength{\parskip}{2.0mm}
\setlength{\parindent}{0mm}

% Questionbox commands:
\newcounter{quescount}
\setcounter{quescount}{0}
\newcommand{\quesbox}[2]{\begin{center}\refstepcounter{quescount}\fbox{\parbox{130mm}{
\label{#1}{\bf Q. \arabic{quescount}:} #2} } \end{center} }


% title information:
\title{CPS: Code Development Progress \& Roadmap.}
\author{A.N. Jackson \& S. Booth}
\date{\mbox{\small $$Revision: 1.4 $$  $$Date: 2004/08/18 11:57:35 $$}}

%------------------------------------------------------------------
\begin{document}

\maketitle

\tableofcontents
\newpage

%-------------------------------------------------------------------
\section{Code development: Work Completed So Far}

\subsection{4.0.0 Columbia [2001/06/15]}
The original version of the code recieved by EPCC from Columbia.

\ukqcdtag{phys\_4\_0\_0\_open}

\subsection{4.1.0  UKQCD [2002/03/06]}
ANSIfication, initial MPI SCU, type flexibility (float/double), basic qcdio
support includuing the qload guage configuration loader.  Off-QCDSP
compatability for the make structure and the regression testing.  Modified the
test suite to streamline the required output.

\ukqcdtag{phys\_4\_1\_0\_alpha\_open}

\ukqcdtag{phys\_4\_1\_0\_beta\_open}

\ukqcdtag{Root-of-Columbia4\_1\_1\_test}

\subsection{4.1.0 Columbia [2002/01/21]}
Integration of new physics components at Columbia.

\ukqcdtag{Columbia4\_1\_1\_test-branch}

\subsection{4.1.1 Assimilation [2002/03/11]}
UKQCD \& Columbia 4.1.0 code merge.

\ukqcdtag{Merged-from-Columbia4\_1\_1\_test}

\subsection{4.1.2 UKQCD Gauge Configuration I/O [2002/05/31]}
Added routines to load and save UKQCD-format gauge configurations into and out
of the CPS.  Also the qcdio test program now creates dummy data and valudates
the data recieved by the processor nodes, and well as producing output files
that are bitwise-identical to the original data.

\ukqcdtag{phys\_4\_1\_2}

%-------------------------------------------------------------------
\section{Code development: Work Outstanding}

\subsection{4.2.0 Hypercube RNG}
Assimilation of the hypercube RNG, with initial cross-platform testing complete.

\subsection{New Code Structure}
A unified build system, dealing with cross-platform portability and basic 
cross-platform testing.

\subsection{Testing}
Ensure a sensible subset of the test programs can be run easily and
automatically tested against expected results.

\subsection{Assimilation of Brookhaven Code}
Code modifications from Brookhaven.

\subsection{QMP Compatability}
Serial and parallel (SCU) implementations of the QMP.

\subsection{OS Standardization}
Ensuring that the code uses the QMP standard for the communications calls.



%-------------------------------------------------------------------
\end{document}

